% Einbindung der Konfigurationsdatei
%===============================================================================
% zentrale Layout-Angaben und Befehle
%===============================================================================
%
\RequirePackage{ifpdf}
\ifpdf
\documentclass[pdftex, a4paper, 12pt]{article}
\else
\documentclass[a4paper, 12pt]{article}
\fi
\usepackage[utf8]{inputenc}
\usepackage{fancyhdr}
\usepackage[T1]{fontenc}
\usepackage{ae}
\usepackage{color}
\usepackage[printonlyused]{acronym}
%
\ifpdf
\usepackage[pdftex,bookmarksopen,bookmarksnumbered]{hyperref}
\usepackage[pdftex]{graphicx}
\pdfcompresslevel=9
\else
\usepackage{url}
\usepackage[dvips]{graphicx}
\fi
%
% ausführlichere Fehlermeldungen
\errorcontextlines=999
%
% Page-Layout
\setlength\headheight{14pt}
\setlength\topmargin{-15,4mm}
\setlength\oddsidemargin{-0,4mm}
\setlength\evensidemargin{-0,4mm}
\setlength\textwidth{160mm}
\setlength\textheight{252mm}
%
% Absatzeinstellungen
\setlength\parindent{0mm}
\setlength\parskip{2ex}
%
% Anweisung zur Erstellung der Titelseite
% #1 = Name des Seminars
% #2 = Name der Ausarbeitung
% #3 = Autor
% #4 = Betreuer
% #5 = Semester, z.B. SoSe 2002
% #6 = {Bachelor|Master|Diplom}
\renewcommand{\maketitle}[6] {
\pagenumbering{Alph}
  \begin{titlepage}
  \centering
    \begin{minipage}[t]{16cm}
      \begin{minipage}{3cm}
        \includegraphics[height=26mm]{include/logo}
      \end{minipage}
      \hfill
      \begin{minipage}{9cm}
        \centering
        University of Bamberg\\[12pt]
        {\Large Distributed Systems Group}
      \end{minipage}
      \hfill
      \begin{minipage}{3cm}
        \includegraphics[height=26mm]{include/UB-Logo-neu_blau-cmyk}
      \end{minipage}
    \end{minipage}\\[108pt]
    {\LARGE Seminar paper}\\[18pt]
    As part of the #6 seminar\\[12pt]
    {\Large\bf #1}\\[92pt]
    With the subject:\\[24pt]
    {\Huge #2}\\
    \vfill
    \begin{minipage}{\textwidth}
      \center
      Submitted by:\\
      {\Large #3\\[18pt]}
      Supervisor: #4\\[12pt]
      Bamberg, #5
    \end{minipage}
  \end{titlepage}
}
%
% setzen nur von fusszeile
\newcommand{\setbottom}
{
\pagestyle{fancy}
\fancyhf{}
\fancyfoot[LO]{\footnotesize\sc Distributed Systems Group}
\fancyfoot[RO]{\thepage}
\renewcommand{\headrulewidth}{0pt}
\renewcommand{\footrulewidth}{0pt}
}
%
% Einbindung eines Bildes
% #1 = label für \ref-Verweise
% #2 = Name des Bildes ohne Endung relativ zu images-Verzeichnis
% #3 = Beschriftung
% #4 = Breite des Bildes im Dokument in cm
\newcommand{\bild}[4]{%
  \begin{figure}[htb]%
    \begin{center}%
      \includegraphics[width=#4cm]{images/#2}%
      \vskip -0.3cm%
      \caption{#3}%
      \vskip -0,2cm%
      \label{#1}%
    \end{center}%
  \end{figure}%
}
%
% Umgebung für Fliesstext um Grafik
% #1 = Ausrichtung: r, l, i, ...
% #2 = Breite des Bildes in cm
% #3 = Name des Bildes ohne Endung relativ zu images-Verzeichnis
% #4 = Beschriftung
% #5 = label für \ref-Verweise
\newcommand{\fliesstext}[5]{%
\begin{wrapfigure}{#1}{#2cm}%
\includegraphics[width=#2cm]{images/#3}%
\caption{#4}%
\label{#5}%
\end{wrapfigure}%
}
%

%
\begin{document}
%
% Titelblatt erstellen
\maketitle{Bachelor|Masterarbeit}{Wirtschaftsinformatik}{Thema der Arbeit}{Autor}{dd.mm.yyyy}
%
% Erstellung der Inhaltsverzeichnisse
\pagenumbering{Roman}
\tableofcontents
\newpage
\listoffigures
\newpage
\listoftables
\newpage
\lstlistoflistings
\newpage
\fancyhead[LO]{\footnotesize\sc\nouppercase{Abk�rzungsverzeichnis}}
%
% Abkürzungen
%
\section*{Abkürzungsverzeichnis}
% Keep this ABC sorted
% In Klammern steht das längste Akronym!
\begin{acronym}[LSPI]
	\acro{API}{Application Programming Interface}
	\acro{CLI}{Command Line Interface}
	\acro{DSG}{Distributed Systems Group}
	\acro{JVM}{Java Virtual Machine}
	\acro{LSPI}{Lehrstuhl für Praktische Informatik}
\end{acronym}
\newpage
\fancyhead[LO]{\footnotesize\sc\nouppercase{\leftmark}}
\setcounter{page}{1}
\pagenumbering{arabic}
%
% Hier einzelne Kapitel mit \input{sections/Kapitel-File} einf�gen.
% Bei gr��eren Arbeiten empfiehlt es sich, zumindest nach Kapiteln oder
% noch genauer aufzuteilen!
%
%
\section{Einleitung}\label{sec:Einleitung}
%
Hier folgt der Text der Ausarbeitung. 
Und noch ein Tipp: Die Seminarvorlage der \acs{DSG} mit Anleitung ist f�r Details sicher hilfreich.

% usw.

\newpage
%
\section{Ausleitung}\label{sec:Ausleitung}
%
Hier befindet sich das letzte Kapitel der Arbeit.
Anschließend noch das Literaturverzeichnis mit Referenzen, wie \cite{tur38} und die Eigenständigkeitserklärung.
%

%
\newpage
%
% Einstellungen f�r Literaturverzeichnis
\addcontentsline{toc}{section}{\bibname}
\bibliographystyle{geralpha}
\selectlanguage{german}
%
% Hier das Literaturverzeichnis einbinden
\bibliography{bibliography/references}
\newpage
%
% Eigenst�ndigkeitserkl�rung
\makedeclaration{Masterarbeit}{16.11.2012}{Alan Turing}
%
\end{document}
