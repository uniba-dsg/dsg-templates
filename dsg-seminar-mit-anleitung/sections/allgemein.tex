%
\section{Genereller Aufbau}
%
Die zentrale Datei der Vorlage ist die Datei \texttt{seminar.tex}, die vom Autor f�r jeweilige Arbeit
anzupassen ist. In dieser Datei k�nnen die Einstellungen f�r die Titelseite vorgenommen
werden. Des Weiteren m�ssen in dieser Datei die einzelnen Kapitel der Arbeit, sowie die
\textit{.bib}-Datei mit den \textit{bibtex}-Eintr�gen eingebunden werden.
%
\subsection{Erstellung einer Titelseite}
%
Die Erstellung der Titelseite erfolgt �ber den Befehl \texttt{\textbackslash maketitle}. Die Anzahl der Parameter,
mit der dieser Befehl aufgerufen wird, h�ngt davon ab, ob eine Seminararbeit oder
eine Bachelor-/Masterarbeit erstellt werden soll.
Zur Erstellung einer Titelseite f�r eine Seminararbeit muss der Befehl \texttt{\textbackslash maketitle} mit
sechs Parametern aufgerufen werden.
%
\begin{verbatim}
\maketitle{Thema des Seminars}{Thema der Ausarbeitung}%
{Autor}{Betreuer}{Semester}{Bachelor|Master}
\end{verbatim}
%
Die einzelnen Paramter sind selbsterkl�rend. Falls die Seminararbeit von mehreren Autoren
verfasst wird, sind diese durch \texttt{,\textbackslash\textbackslash} zu trennen. Bei drei Autoren ist also z.B. die
Zeichenkette \texttt{Hans Meier,\textbackslash\textbackslash\ Peter M�ller und\textbackslash\textbackslash\ Hans M�ller} als dritten Parameter
einzusetzen. Als Semesterangabe sollte ein K�rzel wie \texttt{SoSe13} �bergeben werden.

Um die Titelseite einer Abschlussarbeit zu erstellen, ist der Befehl \texttt{\textbackslash maketitle} mit drei
Parametern aufzurufen.
%
\begin{verbatim}
\maketitle{Bachelor-/Masterarbeit}{Studiengang}{Thema der Arbeit}%
{Autor}{Abgabedatum}
\end{verbatim}
%
Auch in diesem Fall sind die Parameter selbsterkl�rend.
%
\subsection{Einbindung der einzelnen Kapitel}
%
Um die �bersichtlichkeit der erstellten Arbeit zu gew�hrleisten, ist es sinnvoll, f�r jedes
Kapitel eine einzelne \textit{.tex}-Datei zu erstellen. 
Damit unter Verwendung dieser Dateien ein einzelnes Dokument erstellt wird, sind die einzelnen Kapitel in der Datei \texttt{seminar.tex}
einzubinden. Dies erfolgt �ber den Befehl \texttt{\textbackslash input}. Soll z.B. ein Kapitel eingef�gt werden,
dass in der Datei \texttt{kapitel-1.tex} enthalten ist, so geschieht dies durch die folgende
Anweisung.

\texttt{\textbackslash input\{kapitel-1\}}

Diese Anweisung ist an der in der Datei \texttt{seminar.tex} markierten Stelle einzuf�gen. 
%
\subsection{Einbindung des Literaturverzeichnisses}
%
Die einzelnen Literaturquellen sind in einer \textit{.bib}--Datei aufzuf�hren. Die \textit{bibtex}--Datei \texttt{example.bib} enth�lt einige Beispiele, wie
verschiedenen Quellen wie Buch, Artikel usw. in dieser Datei aufzuf�hren sind. Jeder der
Eintr�ge ben�tigt zur Referenzierung ein eindeutiges Label, das sich aus den ersten drei
Buchstaben des Nachnamen des Autors und den letzten beiden Ziffern der Jahreszahl der
Ver�ffentlichung ergeben. Sind mehrere Autoren bei einer Literaturquelle aufgef�hrt, so
ergibt sich der Label aus den Anfangsbuchstaben der Nachnamen der ersten drei Autoren
und den letzten beiden Ziffern der Jahreszahl der Ver�ffentlichung. Falls ein Autor in einem
Jahr mehrere Werke ver�ffentlich hat, so sind die weiteren Label mit kleinen Buchstaben,
beginnend bei \textit{b}, zu erg�nzen.

Eine \textit{.bib}-Datei ist mit Hilfe der Anweisung \texttt{\textbackslash bibliography} in der Datei \texttt{seminar.tex} einzubinden.
Die hier verwendete Datei \texttt{references.bib} wurde also durch durch folgende Anweisung
in dieses Dokument eingebunden.

\texttt{\textbackslash bibliography\{references\}}