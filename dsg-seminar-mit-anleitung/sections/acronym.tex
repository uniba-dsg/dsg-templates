%
\section{Nutzung des Abkürzungsverzeichnisses}
%
Zur Einbindung des Abkürzungsverzeichnisses wird das \texttt{acronym}--Package\footnote{\url{http://www.ctan.org/tex-archive/macros/latex/contrib/acronym}} verwendet.
Der Eintrag aller verwendeten Abkürzungen erfolgt in der Datei \texttt{abbreviations.tex}, die bspw. den folgenden Inhalt hat:
%
\begin{verbatim}
	\begin{acronym}[LSPI]
	 \acro{DSG}{Distributed Systems Group}
	 \acro{LSPI}{Lehrstuhl für Praktische Informatik}
	\end{acronym}
\end{verbatim}
%
Die geklammerte Abkürzung \texttt{[LSPI]} sollte dabei durch die längste vorhandene Abkürzung ersetzt werden.
Eine Abkürzung wird mit dem Befehl \texttt{\textbackslash acro} innerhalb der \texttt{acronym}--Umgebung definiert.
Standardmäßig werden nur die im Text tatsächlich verwendeten Abkürzungen auch im Verzeichnis ausgegeben.
Um eine Abkürzung innerhalb eines Textes einzufügen genügt der folgende Ausdruck:

\texttt{\textbackslash ac\{}\textit{<acronym>}\texttt{\}}

Bei der ersten Verwendung innerhalb des Textes bewirkt die Anweisung, dass der vollständige Name gefolgt von der Abkürzung in Klammern ausgegeben wird.
Bei jeder weiteren Verwendung wird lediglich die Kurzform ausgegeben. Die Anweisung \texttt{\textbackslash ac\{DSG\}} erzeugt also zunächst die Langform "`\ac{DSG}"'.
Anschließend nur noch die Abkürzung "`\ac{DSG}"'.