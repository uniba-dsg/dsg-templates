%
\section{Einbindung von Java-Quellcode}
%
Zur Einbindung von Java-Quellcode ist das Paket \emph{listings} notwendig, dass
u.U. nachinstalliert werden muss. Dieses Paket erm�glicht es, dass Quellcode von
verschiedenen Sprachen auf eine �bersichtliche Weise in das Dokument eingebunden
werden kann. Die Einbindung von Java-Quellcode kann auf zwei Arten erfolgen: zum
einen kann der Java-Quellcode im Quelltext des Dokumentes aufgelistet werden,
zum anderen kann der Quellcode aus einer Datei eingelesen werden.

Die Einbindung von Quellcode, der im Quelltext aufgef�hrt ist, muss �ber die
Nutzung einer Umgebung geschehen. Dies soll an einem kleinen Beispiel gezeigt
werden.
\begin{verbatim}
\begin{javacode}{Das ist das erste Listing}{listing1}
public class Test{

  public Test(){
    // mach was
  }

}
\end{javacode}
\end{verbatim}
%
Zun�chst wird eine Umgebung mit der Bezeichnung \texttt{javacode}
erstellt. Als Parameter wird die �berschrift, sowie das Label �bergeben. 
In der Umgebung wird der Java-Quellcode in den Quelltext des Dokumentes eingef�gt und anschlie�end
die Umgebung wieder beendet. Der obige Code erzeugt das folgende Listing
\ref{listing1}.

\begin{javacode}{Das ist das erste Listing}{listing1}
public class Test {

  public Test(){
    // mach was
  }

}
\end{javacode}

Alternativ kann Quellcode auch aus einer Datei eingebunden werden. Dies ist
vorteilhaft, weil die �bersichtlichkeit des Quelltextes verbessert wird. Die
Einbindung einer Quellcode-Datei erfolgt �ber den Befehl
\texttt{\textbackslash javacode}. Damit die Datei erfolgreich eingebunden werden kann, muss
sie im Verzeichnis \texttt{src} liegen. Der Befehl kann wie folgt genutzt werden.
\begin{verbatim}
\javafile{src.java}{Das ist das zweite Listing}{listing2}
\end{verbatim}

Als ersten Parameter muss der Name der Datei (mit Endung) im
\emph{src}-Verzeichnis �bergeben werden. Der zweite Parameter bestimmt die
�berschrift des Listings. Der dritte Parameter dient zur Referenzierung. Es wird
der gleiche Quelltext erstellt, wie in Listing \ref{listing2} zu sehen ist.

\javafile{src.java}{Das ist das zweite Listing}{listing2}