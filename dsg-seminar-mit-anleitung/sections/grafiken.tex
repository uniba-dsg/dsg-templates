%
\section{Einbindung von Grafiken}
%
Im Allgemeinen werden Grafiken zwischen Absätzen eingefügt. Eine andere
Möglichkeit ist es, die Grafiken von Text umfließen zu lassen. Generell gilt
aber, dass die Grafiken als \emph{pdf}- oder \emph{eps}-Dateien vorliegen
sollten. Dies hängt davon ab, ob \emph{latex} oder \emph{pdflatex} zur Erzeugung
des Dokumentes verwendet wird. Wird \emph{latex} zur Erzeugung genutzt, so
können nur \emph{eps}-Dateien eingebunden werden. Wird das Dokument mit
\emph{pdflatex} erstellt, so sind \emph{pdf}-Dateien zu verwenden.

Grafiken müssen generell im \emph{images}-Verzeichnis vorhanden sein, um sie
erfolgreich einbinden zu können. Selbst erstellte Grafiken müssen also in das
\emph{images}-Verzeichnis kopiert werden.

\subsection{Grafiken zwischen Absätzen}
%
Eine Einbindung von Grafiken zwischen Absätzen erfolgt mit dem Befehl
\texttt{\textbackslash asfigure}.
\begin{verbatim}
\asfigure{bild1}{vs-logo}{DSG-Logo}{5}
\end{verbatim}
Der erste Parameter \texttt{bild1} dient als Label für Referenzen auf die
Grafik. Der zweite Parameter ist der Name der Grafikdatei ohne Endung. Als
dritten Parameter ist dem Befehl \texttt{\textbackslash asfigure} die Beschriftung
der Abbildung zu übergeben. Der letzte Parameter gibt
die Breite der Grafik in cm an. Die Größenangabe darf aber nur als Zahl erfolgen.
Ist eine Grafik breiter als der übergebene Parameter, so wird sie verkleinert, kleinere Grafiken werden
entsprechend vergrößert. Der obige Befehl führt zu der folgenden Abbildung
\ref{bild1}.

\asfigure{bild1}{vs-logo}{DSG-Logo}{5}

\subsection{Von Text umflossene Grafiken}
%
Sollen Grafiken von Text umflossen werden, so ist der Befehl
\texttt{\textbackslash textflow} zu benutzen. Dieser Befehl basiert auf dem Paket
\emph{wrapfig}, dass u.U. nachinstalliert werden muss. Dieses
\textflow{r}{3}{vs-logo}{DSG-Logo}{bild2}
Paket ist, wie alle anderen Umgebungen mit dieser Funktionalität, etwas
beschränkt und zum Teil fehlerhaft, deshalb ist die Nutzung nur unter Vorbehalt
empfohlen. Eine Grafik kann wie folgt eingebunden werden.\\\\
\verb+\textflow{r}{3}{vs-logo}{DSG-Logo}{bild2}+\\\\
Nach diesem Befehl folgt der Text, der das eingebundene Bild umfliessen
soll. Für den ersten Parameter kommen als einzig relevante Möglichkeiten die
kleinen Buchstaben \emph{l} und \emph{r} in Frage. Diese sorgen für eine
Anordnung der Grafik auf der linken, bzw. rechten Seite. Mit Hilfe des zweiten
Parameters wird die gewünschte Breite der Grafik in cm angegeben, die
entsprechend skaliert wird. Der dritte Parameter ist der Name der Grafikdatei
ohne Endung. Der vierte Parameter bestimmt die Bezeichnung der Grafik. Durch den
fünften Parameter kann ein Label zur Referenzierung angegeben werden.

Eine Eigenheit dieses Pakets ist es, dass Grafiken immer am Anfang eines
Absatzes eingefügt werden. Soll eine Grafik von einem Absatz komplett
umschlossen werden, wie dies bei der Abbildung \ref{bild2} der Fall ist, so ist
der \texttt{\textbackslash fliesstext}-Befehl nicht vor dem Absatz aufzuführen, sondern in
den Text zu integrieren. Der obige \texttt{\textbackslash fliesstext}-Befehl ist demnach
hinter dem Wort "`Dieses"' in den Quelltext eingefügt worden.
%