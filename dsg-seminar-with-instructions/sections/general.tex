%
\section{General structure}
%
The central file of the template is \texttt{seminar.tex} which needs to be adjusted by the specific author for a specific thesis.
In the file, settings for the title page can be changed.
Furthermore, the files for the individual sections need to be included, as well as the \textit{.bib}-file containing \textit{bibtex}-entries.
%
\subsection{Creation of a title page}
%
A title page can be created using the command \texttt{\textbackslash maketitle}. 
Depending on whether it is a seminar paper or a Bachelor/Masterthesis, the command has a different number of parameters. 
To create a title page for a seminar paper, six parameters need to be passed to \texttt{\textbackslash maketitle}.
%
\begin{verbatim}
\maketitle{Seminar topic}{Paper topic}%
{Author}{Supervisor}{Semester}{Bachelor|Master}
\end{verbatim}
%
The individual parameters are self-explanatory.
If the seminar paper is written by more than one author, the names should be separated by \texttt{,\textbackslash\textbackslash}.
In the case of three authors, the third parameter would therefore be  \texttt{Hans Meier,\textbackslash\textbackslash\ Peter Müller and\textbackslash\textbackslash\ Hans Müller} as an example.
For the semester, an abbreviation should be used, such as \texttt{SoSe19}.

To create a title page for a Bachelor/Masterthesis, the command \texttt{\textbackslash maketitle} has to be called with three parameters.
%
\begin{verbatim}
\maketitle{Bachelor-/Masterthesis}{Course of studying}{Thesis topic}%
{Author}{Submission date}
\end{verbatim}
%
Also in this case, the parameters are self-explanatory.
%
\subsection{Including sections}
%
To keep the work well-arranged, it is reasonable to create a \textit{.tex}-file for each section.
To assemble the files to one single document, the sections need to be included using the command  \texttt{\textbackslash input} inside the file \texttt{seminar.tex}.
If, for example, a section should be included which is inside the file \texttt{section-1.tex}, then the following command would be used:

\texttt{\textbackslash input\{section-1\}}

%
\subsection{Including the bibliography}
%
The literature sources should be listed in a \textit{.bib}--file.
The \textit{bibtex}--file \texttt{example.bib} contains some examples of different sources, like books, articles, etc.
Each entry needs a unique label, which is formed from the first three letters of the lastname of the author and the last two digits of the year when it was published.
If there is more than one author for a source, the label is formed from the beginning letters of the lastnames of the first three authors and the last two digits of the year when it was published.
If an author has published more than one work within a year, the following labels should be appended with small letters, starting with \textit{b}.

A \textit{.bib}-file should be included using \texttt{\textbackslash bibliography} within the file \texttt{seminar.tex}.
The exemplary file \texttt{references.bib} was therefore included using:

\texttt{\textbackslash bibliography\{references\}}