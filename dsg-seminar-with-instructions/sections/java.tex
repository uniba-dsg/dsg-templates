%
\section{Including Java source code}
%
To include Java source code, the package \emph{listings} is required, which might have to be installed.
This package enables the inclusion of source code from various programming languages into the document.
There are two possibilities to include source code in a document. 
One possibility is to put the source code directly in the document and the other is to read it from a file.

To include source code directly, a special environment is required which is shown in the following example:

\begin{verbatim}
\begin{javacode}{This is the first listing}{listing1}
public class Test{

  public Test(){
    // do something
  }

}
\end{javacode}
\end{verbatim}
%
First, an environment called \texttt{javacode} is created.
Required parameters are a caption and a label.
Then follows the actual source code and the environment is closed again.
The above example leads to listing \ref{listing1}.

\begin{javacode}{This is the first listing}{listing1}
public class Test {

  public Test(){
    // do something
  }

}
\end{javacode}

As an alternative, the source code can also be read from a file.
This is beneficial, because it increases the readability.
To include source code from a file, the command \texttt{\textbackslash javafile} is used. 
For a successful inclusion, the source code file has to be located in the \texttt{src}-directory. 

Example usage:
\begin{verbatim}
\javafile{src.java}{This is the second listing}{listing2}
\end{verbatim}

The first parameter has to be the name of the file (with file ending) inside the \emph{src}--directory.
The second parameter defines the caption and the third parameter is the label for referencing the listing.

The above example leads to listing \ref{listing2} as a result.

\javafile{src.java}{This is the second listing}{listing2}