%
\section{Introduction}
%
For writing theses, the Distributed Systems Group provides LaTeX templates.
This document illustrates how the specific environments and commands of the template can be used for creating theses.
However, this is no general guide on creating documents with \LaTeX.
As an introductory literature, the book by Kopka \cite{Kopka2003} is recommended.
There are other older books which provide a good introduction as well.
A good overview is also provided by the wikibook "`\LaTeX"' \cite{lat12}.
Furthermore, a lot of information can be found online using various search engines.

A \LaTeX-document can be created by running \texttt{pdflatex seminar} or by using an appropriate IDE, like \textit{TeXnicCenter}\footnote{\url{http://www.texniccenter.org}} or \textit{TeXlipse}\footnote{\url{http://texlipse.sourceforge.net}}.
In addition, the tool SumatraPDF\footnote{\url{https://www.sumatrapdfreader.org/free-pdf-reader.html}} can be used which enables a forward and backward search inside the documents.