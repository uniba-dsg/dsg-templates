%
\section{Using the list of abbreviations}
%
To include the list of abbreviations, the package \texttt{acronym}\footnote{\url{http://www.ctan.org/tex-archive/macros/latex/contrib/acronym}} is used.
Used abbreviations are added to the \texttt{abbreviations.tex}--file, which can for example have the following content:
%
\begin{verbatim}
	\begin{acronym}[LSPI]
	 \acro{DSG}{Distributed Systems Group}
	 \acro{LSPI}{Lehrstuhl für Praktische Informatik}
	\end{acronym}
\end{verbatim}
%
The abbreviation in square brackets \texttt{[LSPI]} should be the longest of all used abbreviations. 
An abbreviation can be created with the command \texttt{\textbackslash acro} inside the \texttt{acronym}--environment.
As a default, only the abbreviations which are actually used in the text are included in the list of abbreviations.
To use an abbreviation inside the text, the following expression is sufficient:

\texttt{\textbackslash ac\{}\textit{<acronym>}\texttt{\}}

When an abbreviation is used for the first time, this expression leads to the abbreviation being written out followed by the abbreviation in brackets.
Afterwards, only the abbreviation is printed.
The command \texttt{\textbackslash ac\{DSG\}} therefore creates the written out version when it is used for the first time: "`\ac{DSG}"'.
And for every following usage, only the abbreviation is printed: "`\ac{DSG}"'.