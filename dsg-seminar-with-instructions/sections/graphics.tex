%
\section{Including graphics}
%
Generally, graphics are included between paragraphs.
Another option is to surround graphics with text.
Graphics should be provided as or \emph{pdf} or \emph{eps}-files.
This depends on whether \emph{latex} or \emph{pdflatex} is used to create the document.
If \emph{latex} is used, only \emph{eps}-files can be included.
If the document is created with \emph{pdflatex}, \emph{pdf}-files should be used.

Graphics should be inside the \emph{images}-directory, to easily include them. 
Newly created graphics should therefore be copied to the \emph{images}-directory.

\subsection{Graphics between paragraphs}
%
Graphics can be included between paragraphs using 
\texttt{\textbackslash bild}.
\begin{verbatim}
\bild{picture1}{vs-logo}{DSG-Logo}{5}
\end{verbatim}
The first parameter \texttt{picture1} serves as a label to reference the graphic. 
The second parameter is the name of the graphics file without its file ending.
As a third parameter, the caption has to be provided and the last parameter defines the width of the graphic in centimeters. 
The width has to be provided as a numeric value. 
If a graphic is wider than the provided width, it is shrinked.
Smaller graphics are scaled up.
The above command leads to figure \ref{picture1}.

\bild{picture1}{vs-logo}{DSG-Logo}{5}

\subsection{Graphics surrounded by text}
%
If a graphic should be surrounded by text, the command \texttt{\textbackslash fliesstext} has to be used.
This command is based on the package \emph{wrapfig} which might have to be installed.
\fliesstext{r}{3}{vs-logo}{DSG-Logo}{picture2}
This package is, similar to all other environments with this functionality, limited and partially erroneous.
Therefore it should be used only with caution.
A graphic can be included with:\\\\
\verb+\fliesstext{r}{3}{vs-logo}{DSG-Logo}{picture2}+\\\\
After this command, the text follows which should surround the figure.
For the first parameter, only the small letters \emph{l} and \emph{r} are possible.
They lead to the graphic being arranged on the left or right side.
The second parameter can be used to specify the desired width in centimeters.
The graphic is scaled accordingly.
The third parameter is the name of the graphics file without its file ending.
The fourth parameter defines the caption for the figure and with the fifth parameter, a label can be defined to reference the figure.

A peculiarity of this package is that graphics are always placed at the beginning of a paragraph.
If a graphic should be enclosed completely by a paragraph, as it is the case with figure \ref{picture2}, the command 
\texttt{\textbackslash fliesstext} should not be placed before the paragraph but within it.
The above command is therefore placed after the word "`installed"'.
%